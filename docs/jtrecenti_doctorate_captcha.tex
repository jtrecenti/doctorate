\documentclass[12pt,]{report}
\usepackage{lmodern}
\usepackage{amssymb,amsmath}
\usepackage{ifxetex,ifluatex}
\usepackage{fixltx2e} % provides \textsubscript
\ifnum 0\ifxetex 1\fi\ifluatex 1\fi=0 % if pdftex
  \usepackage[T1]{fontenc}
  \usepackage[utf8]{inputenc}
\else % if luatex or xelatex
  \ifxetex
    \usepackage{mathspec}
  \else
    \usepackage{fontspec}
  \fi
  \defaultfontfeatures{Ligatures=TeX,Scale=MatchLowercase}
\fi
% use upquote if available, for straight quotes in verbatim environments
\IfFileExists{upquote.sty}{\usepackage{upquote}}{}
% use microtype if available
\IfFileExists{microtype.sty}{%
\usepackage[]{microtype}
\UseMicrotypeSet[protrusion]{basicmath} % disable protrusion for tt fonts
}{}
\PassOptionsToPackage{hyphens}{url} % url is loaded by hyperref
\usepackage[unicode=true]{hyperref}
\PassOptionsToPackage{usenames,dvipsnames}{color} % color is loaded by hyperref
\hypersetup{
            pdftitle={Quebrando CAPTCHAs},
            pdfauthor={Julio Trecenti},
            colorlinks=true,
            linkcolor=Maroon,
            citecolor=Blue,
            urlcolor=Blue,
            breaklinks=true}
\urlstyle{same}  % don't use monospace font for urls
\usepackage{natbib}
\bibliographystyle{apalike}
\usepackage{longtable,booktabs}
% Fix footnotes in tables (requires footnote package)
\IfFileExists{footnote.sty}{\usepackage{footnote}\makesavenoteenv{long table}}{}
\IfFileExists{parskip.sty}{%
\usepackage{parskip}
}{% else
\setlength{\parindent}{0pt}
\setlength{\parskip}{6pt plus 2pt minus 1pt}
}
\setlength{\emergencystretch}{3em}  % prevent overfull lines
\providecommand{\tightlist}{%
  \setlength{\itemsep}{0pt}\setlength{\parskip}{0pt}}
\setcounter{secnumdepth}{5}
% Redefines (sub)paragraphs to behave more like sections
\ifx\paragraph\undefined\else
\let\oldparagraph\paragraph
\renewcommand{\paragraph}[1]{\oldparagraph{#1}\mbox{}}
\fi
\ifx\subparagraph\undefined\else
\let\oldsubparagraph\subparagraph
\renewcommand{\subparagraph}[1]{\oldsubparagraph{#1}\mbox{}}
\fi

% set default figure placement to htbp
\makeatletter
\def\fps@figure{htbp}
\makeatother

\usepackage[brazilian]{babel}
\usepackage[utf8]{inputenc}
\usepackage[T1]{fontenc}
\usepackage{lipsum}
\usepackage{fullwidth}
\usepackage{indentfirst}
\usepackage[left=2.5cm, right=2.5cm, top=4cm, bottom=3.8cm]{geometry}
\renewcommand{\familydefault}{\sfdefault}
\PassOptionsToPackage{dvipsnames}{xcolor}
\RequirePackage{xcolor} % [dvipsnames]
\definecolor{halfgray}{gray}{0.55} % chapter numbers will be semi transparent .5 .55 .6 .0
\definecolor{webgreen}{rgb}{0,.5,0}
\definecolor{webbrown}{rgb}{.6,0,0}
%\definecolor{Maroon}{cmyk}{0, 0.87, 0.68, 0.32}
%\definecolor{RoyalBlue}{cmyk}{1, 0.50, 0, 0}
%\definecolor{Black}{cmyk}{0, 0, 0, 0}
\usepackage{fancyhdr}
\usepackage{pdfpages}
\usepackage{amsmath}
\usepackage{graphicx}
\usepackage{listings}
\usepackage{enumitem}
\usepackage{setspace}
\usepackage{spverbatim}
\usepackage{lipsum}
\usepackage{natbib}
\usepackage{longtable}
\usepackage{booktabs}
\usepackage{background}

% \newcommand{\prestadorEmpresaFoot}{Associação Brasileira de Jurimetria}
% \newcommand{\prestadorEmpresa}{Associação Brasileira de Jurimetria}
% \newcommand{\prestadorRepr}{Marcelo Guedes Nunes}
% \newcommand{\prestadorEnderecoFoot}{Rua Gomes de Carvalho, 1356, 1º andar. CEP 04547-005 - São Paulo, SP, Brasil.}
% \newcommand{\prestadorEndereco}{Rua Gomes de Carvalho, 1356, 1º andar}
% \newcommand{\prestadorEnderecoComp}{CEP 04547-005 - São Paulo, SP, Brasil}
% \newcommand{\prestadorSite}{\url{http://abj.org.br}}
% \newcommand{\prestadorEmail}{contato@abj.org.br}
% \newcommand{\logo}{\includegraphics[width=0.21\textwidth, trim=0cm 0cm 0cm 11.1cm, clip]{imgs/logo_abj.png}}


\setlength{\parindent}{2em}

% \backgroundsetup{
% scale=1,
% angle=0,
% opacity=1,
% color=black,
% contents={\begin{tikzpicture}[remember picture,overlay]
% \node at ([xshift=-4.35cm,yshift=-2.5cm] current page.north east) % Adjust the position of the logo.
% {\logo}; % logo goes here
% \end{tikzpicture}}
% }

\usepackage{float}
\let\origfigure\figure
\let\endorigfigure\endfigure
\renewenvironment{figure}[1][2] {
    \expandafter\origfigure\expandafter[H]
} {
    \endorigfigure
}

\title{Quebrando CAPTCHAs}
\author{Julio Trecenti}
\date{03 de maio de 2018}

\begin{document}
\maketitle

{
\hypersetup{linkcolor=black}
\setcounter{tocdepth}{2}
\tableofcontents
}
\listoftables
\listoffigures
\chapter{Introduction}\label{introduction}

\chapter{Introdução}\label{introducao}

\section{Objetivos}\label{objetivos}

\section{Resultados Esperados}\label{resultados-esperados}

\section{Organização do trabalho}\label{organizacao-do-trabalho}

\chapter{Problema}\label{problema}

O problema do Captcha pode ser entendido como um problema de
classificação de imagens. Especificamente, nosso interesse é criar uma
função \(g\) que recebe uma imagem
\(\mathbf X = \{x_{nmr} \in [0,1]\}_{N\times M \times R}\) e retorna um
vetor de índices \(\mathbf y\), sendo que cada índice \(y_j\)
corresponde a um caractere \(c_j\), \(j = 1, \dots, L\), onde \(L\) é o
número de caracteres contidos na imagem.

Das afirmações anteriores podemos tirar três conclusões.

\begin{enumerate}
\def\labelenumi{\arabic{enumi}.}
\tightlist
\item
  Nossa variável \textbf{explicativa}, a imagem, é uma matriz
  \(\mathbf X = \{x_{ijk}\}_{N\times M \times R}\), em que \(N\) é o
  número de linhas, \(M\) é o número de colunas e \(R\) é o número de
  \emph{cores}, ou \emph{canais}.
\end{enumerate}

O elemento \(x_{nm\cdot}\) é denominado \emph{pixel}. Um pixel
representa a menor unidade possível da imagem. Em uma imagem colorida,
por exemplo, temos \(R=3\). Nesse caso, um pixel é um vetor de três
dimensões com valores entre zero e um, representando a intensidade de
vermelho, verde e azul da coordenada \(n,m\) da imagem. Numa imagem em
escala de cinza, temos \(R=1\) e o pixel, de uma dimensão, representa a
intensidade do cinza (com 1=branco e 0=preto).

\begin{enumerate}
\def\labelenumi{\arabic{enumi}.}
\setcounter{enumi}{1}
\item
  O objeto \(C \in \mathcal A^L\) é um vetor de itens de um alfabeto
  \(\mathcal A\) com tamanho \(|\mathcal A|\), finito e conhecido. Esse
  alfabeto contém todos os possíveis caracteres que podem aparecer na
  imagem.
\item
  Nossa \textbf{resposta}
  \(\mathbf y \in \mathbb \{1, \dots, |\mathcal A|\}^L\) é um vetor de
  índices de tamanho fixo. Cada elemento de \(\mathbf y\) representa um
  valor do alfabeto \(\mathcal A\).
\end{enumerate}

A construção de uma função \(g\) capaz de mapear \(\mathbf y\) a partir
de uma nova imagem \(\mathbf X\) depende de uma amostra de imagens
\(\mathbf X_1, \dots, \mathbf X_S\) corretamente classificadas por
\(\mathbf y_1, \dots, \mathbf y_S\). A tarefa é, portanto, estimar uma
função \(\hat g\) com o objetivo de minimizar.

\[
L(g(\mathbf X), \mathbf y) = \mathbb I(g(\mathbf X) \neq \mathbf y)
\]

em que \(\mathbb I\) é a função indicadora.

\section{Variantes}\label{variantes}

\subsection{Áudio}\label{audio}

\subsection{Covariáveis e número de respostas
variável}\label{covariaveis-e-numero-de-respostas-variavel}

\subsection{reCaptcha}\label{recaptcha}

\chapter{Solução}\label{solucao}

\chapter{Resultados}\label{resultados}

\chapter{Considerações finais}\label{consideracoes-finais}

\chapter{Pacote decryptr}\label{pacote-decryptr}

\chapter{CAPTCHAs em áudio}\label{captchas-em-audio}

\bibliography{bibliography/book.bib,bibliography/packages.bib}

\end{document}
